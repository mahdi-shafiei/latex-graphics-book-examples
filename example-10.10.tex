\documentclass{article}

%% Denote paragraphs with vertical space rather than indenting (not critical)
\usepackage{parskip}

%% Support for URL in introductory text (not needed for main example)
\usepackage{url}

%% *** Enable PGFPLOTS (automatically enables TikZ) ***
\usepackage{pgfplots}

%% Prevent some PGFPLOTS messages (not critical)
\pgfplotsset{compat=1.18,compat/show suggested version=false}


\begin{document}

%% Introductory Text
Example 10.10 from the book\\
\emph{Unlocking LaTeX Graphics: A Concise Guide to Ti$k$Z/PGF and PGFPLOTS}.\\
For more information, visit \url{https://latex-graphics.com}.
\par\bigskip

%% *** START OF EXAMPLE CODE ***
\pgfplotsset{every axis/.style={scale only axis, width=4cm, height=1.5cm}}
\begin{tikzpicture}
  \begin{axis}[name=A,title=Dataset A]
    \addplot[blue,mark=square*] table[y=A]{data.dat};
  \end{axis}
  \path (A.right of south east) ++(0.5,0) coordinate (BB);
  \begin{axis}[name=B,anchor=left of south west,at=(BB),title=Dataset B]
    \addplot[red,mark=*] table[y=B]{data.dat};
  \end{axis}
  \path (A.below south west) ++(0,-0.5) coordinate (CC);
  \begin{axis}[name=C,anchor=above north west,at=(CC),title=Dataset C]
    \addplot[orange,mark=halfsquare*] table[y=C]{data.dat};
  \end{axis}
  \path (B.west |- C.south) coordinate (DD);
  \begin{axis}[anchor=south west,at=(DD),title=Dataset D]
    \addplot[teal,mark=pentagon*] table[y=D]{data.dat};
  \end{axis}
  % Show the coordinates (just for demonstration)
  \fill[gray] foreach \nd/\loc in {BB/below,CC/above,DD/below left} {
    (\nd) circle(2pt) node[\loc]{\nd}};
\end{tikzpicture}
%% *** END OF EXAMPLE CODE ***

\end{document}
