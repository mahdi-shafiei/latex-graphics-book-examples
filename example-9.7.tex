\documentclass{article}

%% Denote paragraphs with vertical space rather than indenting (not critical)
\usepackage{parskip}

%% Support for URL in introductory text (not needed for main example)
\usepackage{url}

%% *** Enable PGFPLOTS (automatically enables TikZ) ***
\usepackage{pgfplots}

%% Prevent some PGFPLOTS messages (not critical)
\pgfplotsset{compat=1.18,compat/show suggested version=false}

%% *** PGFPLOTS library ***
\usepgfplotslibrary{groupplots}

\begin{document}

%% Introductory Text
Example 9.7 from the book\\
\emph{Unlocking LaTeX Graphics: A Concise Guide to Ti$k$Z/PGF and PGFPLOTS}.\\
For more information, visit \url{https://latex-graphics.com}.
\par\bigskip

%% *** START OF EXAMPLE CODE ***
\begin{tikzpicture}
  \begin{groupplot}[
      group style={
        group size=2 by 2, horizontal sep=1cm, vertical sep=1.5cm,
        xlabels at=edge bottom, ylabels at=edge left,
      },
      scale only axis, width=4cm, height=2cm,
      xlabel=$x$, xmin=0.5,xmax=4.5,xtick distance=1,
      ylabel=$y$, ylabel near ticks,
      ymin=0, ymax=15, ytick distance=3
    ]
    \nextgroupplot[title=Dataset A]
    \addplot[blue,mark=*] table[y=A]{data.dat};
    \nextgroupplot[title=Dataset B]
    \addplot[red,mark=*] table[y=B]{data.dat};
    \nextgroupplot[title=Dataset C]
    \addplot[orange,mark=*] table[y=C]{data.dat};
    \nextgroupplot[title=Dataset D]
    \addplot[teal,mark=*] table[y=D]{data.dat};
  \end{groupplot}
\end{tikzpicture}
%% *** END OF EXAMPLE CODE ***

\end{document}
