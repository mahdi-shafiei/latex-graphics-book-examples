\documentclass{article}

%% Denote paragraphs with vertical space rather than indenting (not critical)
\usepackage{parskip}

%% Support for URL in introductory text (not needed for main example)
\usepackage{url}

%% *** Enable PGFPLOTS (automatically enables TikZ) ***
\usepackage{pgfplots}

%% Prevent some PGFPLOTS messages (not critical)
\pgfplotsset{compat=1.18,compat/show suggested version=false}

%% *** PGFPLOTS library ***
\usepgfplotslibrary{groupplots}
%% *** LaTeX package ***
\usepackage{subcaption,float}


\begin{document}

%% Introductory Text
Example 10.9 from the book\\
\emph{Unlocking LaTeX Graphics: A Concise Guide to Ti$k$Z/PGF and PGFPLOTS}.\\
For more information, visit \url{https://latex-graphics.com}.
\par\bigskip

%% *** START OF EXAMPLE CODE ***
\tikzset{subcap/.style={below,text width=4cm,inner ysep=0em}}
\begin{figure}[H]
  \centering
  \begin{tikzpicture}
    \begin{groupplot}[
        group style={group name=foo, group size=2 by 1},
        scale only axis, width=4cm, height=1.5cm,
        xmin=0.5,xmax=4.5,xtick distance=1,
        ymin=0,ymax=12,ylabel near ticks ]
      \nextgroupplot \addplot[blue,mark=*] table[y=A]{data.dat};
      \nextgroupplot \addplot[red,mark=*] table[y=B]{data.dat};
    \end{groupplot}
    % Adding subcaptions
    \node[subcap] at (foo c1r1.below south)
      {\subcaption{Dataset A}\label{fig:a}};
    \node[subcap] at (foo c2r1.below south)
      {\subcaption{Dataset B}\label{fig:b}};
  \end{tikzpicture}
  \caption{Group plot with subcaptions}
  \label{fig:test}
\end{figure}

In Fig.~\ref{fig:test}, Dataset A is shown in Fig.~\ref{fig:a},
and Dataset B is shown in Fig.~\ref{fig:b}.
%% *** END OF EXAMPLE CODE ***

\end{document}
